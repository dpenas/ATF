\documentclass[a4paper]{article}

\usepackage[utf8]{inputenc}
\usepackage[spanish]{babel}
\usepackage{mathtools}
\usepackage{amsmath}
\usepackage{graphics}
\usepackage{multicol}
\usepackage[T1]{fontenc}
\usepackage{hyperref}


\newcommand{\HRule}{\rule{\linewidth}{0.5mm}}

\title{Resolución de problemas con lógica difusa. Segunda Entrega}


\begin{document}

% Título
	\begin{titlepage}
		\begin{center}

			\HRule \\[0.4cm]
			{ \huge \bfseries Memoria Berger y Checksum Honey}\\[0.4cm]
			\HRule \\[0cm]

			\vspace{1cm}
			\textsc{\Large Arquitecturas tolerantes a fallos}\\[0.5cm]
			\textsc{\Large Curso 2012/2013}\\[0.5cm]
		\end{center}

		\vfill
		\hfill
		\emph{Autor:}
		\vspace{0.5cm}
		\\  
		\vspace{0.1cm}
		\hfill Penas Sabín, Darío \texttt{<dario.penas@udc.es>}\\
		\vspace{0.1cm}

	\end{titlepage}
% Índices
\tableofcontents
\clearpage

\section{Introducción}
	Uno de los errores más comunes en los sistemas informáticos son aquellos causados por los canales de comunicación entre los diferentes componentes de un ordenador o, incluso, entre ordenadores, y que causan que los resultados obtenidos no sean válidos por el cambio de uno o varios bits con respecto del original. Para intentar disminuir este tipo de errores se han generado diferentes técnicas, más o menos redundantes, que intentan disminuir la probabilidad de que se cometan alguno de estos fallos.
	Las aquí comentadas serán el Checksum Honey y los códigos de Berger, cada uno con sus ventajas e inconvenientes.

\section{Código de Berger}



\section{Checksum Honey}

\end{document}
